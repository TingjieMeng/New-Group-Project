\documentclass[12pt,]{article}
\usepackage{lmodern}
\usepackage{amssymb,amsmath}
\usepackage{ifxetex,ifluatex}
\usepackage{fixltx2e} % provides \textsubscript
\ifnum 0\ifxetex 1\fi\ifluatex 1\fi=0 % if pdftex
  \usepackage[T1]{fontenc}
  \usepackage[utf8]{inputenc}
\else % if luatex or xelatex
  \ifxetex
    \usepackage{mathspec}
  \else
    \usepackage{fontspec}
  \fi
  \defaultfontfeatures{Ligatures=TeX,Scale=MatchLowercase}
\fi
% use upquote if available, for straight quotes in verbatim environments
\IfFileExists{upquote.sty}{\usepackage{upquote}}{}
% use microtype if available
\IfFileExists{microtype.sty}{%
\usepackage{microtype}
\UseMicrotypeSet[protrusion]{basicmath} % disable protrusion for tt fonts
}{}
\usepackage[margin=1in]{geometry}
\usepackage{hyperref}
\hypersetup{unicode=true,
            pdftitle={External Debt and Currency Crisis in Emerging Markets},
            pdfauthor={New Group Project},
            pdfborder={0 0 0},
            breaklinks=true}
\urlstyle{same}  % don't use monospace font for urls
\usepackage{graphicx,grffile}
\makeatletter
\def\maxwidth{\ifdim\Gin@nat@width>\linewidth\linewidth\else\Gin@nat@width\fi}
\def\maxheight{\ifdim\Gin@nat@height>\textheight\textheight\else\Gin@nat@height\fi}
\makeatother
% Scale images if necessary, so that they will not overflow the page
% margins by default, and it is still possible to overwrite the defaults
% using explicit options in \includegraphics[width, height, ...]{}
\setkeys{Gin}{width=\maxwidth,height=\maxheight,keepaspectratio}
\IfFileExists{parskip.sty}{%
\usepackage{parskip}
}{% else
\setlength{\parindent}{0pt}
\setlength{\parskip}{6pt plus 2pt minus 1pt}
}
\setlength{\emergencystretch}{3em}  % prevent overfull lines
\providecommand{\tightlist}{%
  \setlength{\itemsep}{0pt}\setlength{\parskip}{0pt}}
\setcounter{secnumdepth}{0}
% Redefines (sub)paragraphs to behave more like sections
\ifx\paragraph\undefined\else
\let\oldparagraph\paragraph
\renewcommand{\paragraph}[1]{\oldparagraph{#1}\mbox{}}
\fi
\ifx\subparagraph\undefined\else
\let\oldsubparagraph\subparagraph
\renewcommand{\subparagraph}[1]{\oldsubparagraph{#1}\mbox{}}
\fi

%%% Use protect on footnotes to avoid problems with footnotes in titles
\let\rmarkdownfootnote\footnote%
\def\footnote{\protect\rmarkdownfootnote}

%%% Change title format to be more compact
\usepackage{titling}

% Create subtitle command for use in maketitle
\newcommand{\subtitle}[1]{
  \posttitle{
    \begin{center}\large#1\end{center}
    }
}

\setlength{\droptitle}{-2em}

  \title{External Debt and Currency Crisis in Emerging Markets}
    \pretitle{\vspace{\droptitle}\centering\huge}
  \posttitle{\par}
    \author{New Group Project}
    \preauthor{\centering\large\emph}
  \postauthor{\par}
      \predate{\centering\large\emph}
  \postdate{\par}
    \date{April 25, 2019}


\begin{document}
\maketitle

\textbf{Members:}

\begin{itemize}
  \itemsep-.6em
  \item Liumin Chen (NetID: lc1077) - email: lc1077@georgetown.edu\
  \item Tingjie  Meng (NetID: tm1305) - email: tm1305@georgetown.edu\
\end{itemize}

\textbf{Github Repository:}
\url{https://github.com/TingjieMeng/New-Group-Project}

\section{Problem Statement}\label{problem-statement}

Debts and credits are crucial both for the expansion of businesses and
for economic development. However, when the growth of debts outpaces the
growth of income, debts become a time bomb triggering crises. There are
two types of debts: domestic debts denominated in the domestic currency
and external debts denominated in foreign currencies. The history of
financial crises tells us that countries with a high level of external
debt are especially vulnerable to capital outflows and thus exchange
rate fluctuations. As Ray Dalio describes in , such an event is called
an inflationary depression.

Classic examples of a deflationary depression are the Tukish currency
crises in the early 2000s and 2018. Since the 1980s, Turkey has been
heavily dependent on foreign investments to fund its huge current
account deficits, especially in the capital-intensive construction
sector. Thus its economy is fragile to any internal or external shock
that can lead to foreign capital outflow on a large scale. In 2001,
foreign investors quickly withdrew capital from Turkey amid political
tensions and tightening fiscal policies by the Turkish government, which
caused the value of Turkish lira to plunge. The history repeats itself.
In 2018, due to domestic instability and tightening monetary policies
from major central banks, Turkey saw a year-over-year decline of foreign
investments by 9,500 basis points. Without sufficient foreign capital to
support its weak economy, the Turkish lira soon began drastically losing
value, leaving many companies unable to serve their foreign currency
debts.

Our project focuses on studying the link between external debt and
currency crisis in Emerging Market Economies (EME). As shown in Turkey's
crises, over-dependence on external financing can be extremely dangerous
for EMEs in the long run. However, EMEs typically face difficulties in
issuing debts in local currency because their local financial markets
are underdeveloped. Thus they are more likely to hold a significant
portion of external debt in the total debt portfolio compared to
developed economies. Specifically, we explore factors that strengthen or
weaken the link between external debt and currency crisis in EMEs and
use the patterns we find to predict the possibility of a currency crisis
happening in future years. Our findings can serve as policy guidance for
external debt management for central bank policymakers in emerging
markets.

\section{Research Design}\label{research-design}

\subsection{dataset}\label{dataset}

A set of time series data covering 46 developing countries from 1995 to
2015 \#\# data source Data sources for this study are as follows: the
World Bank Open Data, the International Monetary Fund (IMF) Open Data,
and the Polity IV Project conducted by the Center for Systemic Peace.
All three data sources are publicly available. The World Bank and IMF
APIs can be installed in R, and the Polity IV Project data can be
conveniently downloaded as a csv package.\\
\#\# variable description Dependent variable: possibility of a currency
crisis happening, defined as a nominal depreciation of 15\% within a
year Independent variable of interest: externaldebt/GNI Control
variable: GDP per capita, level of democracy, US 3-month treasury yield,
foreign reserves/total external debt

\section{Methods}\label{methods}

\subsection{data visualization}\label{data-visualization}

Data visualization is used for getting a general understanding of the
relationships between independent variables and dependent variables. It
is also used as case studies of individual countries that experience a
classic pattern of currency crises. \#\# mixed-effect logit model The
logit model in parameterization form is as follows: Pr(Currency Crisis =
1) = φ(β0 + β1(External debt to GNI)i,t + β2(Polity2)i,t + β3(US 3m
yield)i + β4(lag GDP per capita)i,t + β5(lag reserves to external
debt)i,t + yeari + countryt + εi,t) \#\# machine learning techniques We
use several models including K nearest neighbours, CART, and Random
Forest. After comparison, we choose the one model specification with the
best model fit.


\end{document}
